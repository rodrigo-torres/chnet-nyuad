\documentclass[a4paper,12pt]{article}
%\usepackage{jinstpub}
\usepackage{enumerate}
\usepackage{abstract}
\pdfoutput=1 

%% Language and font encodings
\usepackage{textcomp}
\usepackage[english]{babel}
\usepackage[utf8x]{inputenc}
\usepackage[T1]{fontenc}

\renewcommand{\labelitemi}{$\cdot$}
\renewcommand{\abstractname}{Description} 
\renewcommand{\absnamepos}{empty}

\title{\textbf{XRF Scanner Software Changelog}}
\author{INFN CHNet - NYUAD}
\date{}



\begin{document}
\maketitle

\begin{abstract}
	This change-log outlines the source code modifications to the INFN CHNet XRF scanner's software made at the New York University Abu Dhabi (NYUAD) research branch. In particular, versions 10.0 and above feature modifications to the software's structure that allow for a multi-detector scanner architecture
\end{abstract}

\tableofcontents
\clearpage

\section{XRFMultidet\_v10.0}
\textit{Changelog with respect to versions 9.x}

\begin{enumerate}
	\item Several rearrangements to the GUI for medium resolutions.
	\begin{itemize}
		\item Port assignment and initialization members for the linear stages are now all arranged under the same tab.
		\item All laser-tracking and Arduino port controls are now under the same tab.
		\item Several modifications were made to the status labels default values.
		\item Support for different motor initialization protocols has been removed --- the new mode of operation is to set a default variable in the source code, and to use the initialization parameters for the motor of interest.
	\end{itemize}
	\item The \textit{ScreenDetector} external program has been deprecated. The relevant code for screen resolution detection is now embedded in main.cpp 
	\item Several source modifications to main.cpp
	\begin{itemize}
		\item Dependencies already included in the project's headers have been commented out.
		\item The timeout(int seconds) function has been deprecated as it is not called anywhere in the program.
		\item CAEN kernel modules addition has been deprecated as it is improper syntax. The relevant libraries are automatically loaded into the kernel if the computer is physically connected to the DAC and the appropriate drivers are installed.
	\end{itemize}
	\item Source code changes to X\_init.cpp, Y\_init.cpp and Z\_init.cpp
	\begin{itemize}
		\item New local functions to load M404-8PD and M404-2PD parameters have been introduced so as to cut down on redundant code.
		\item A new global function to move X, Y, and Z motors to their reference positions has been introduced.
		\item Some global and local variables have been refactored to appeal more to English-speaking users.
		\item Junk code has been removed, and the three files have been condensed onto one file name motors\_init.cpp
	\end{itemize}
	\item Several modifications were made throughout the source code to account for the changes specified in the item above.
	\item The file move\_motors.cpp has been renamed to motors\_move.cpp for consistency.
	\item Source code changes to mainwindow.cpp
	\begin{itemize}
		\item X, Y, and Z specific functions for read\_answer\_XXX() and send\_command\_XXX() have been replaced by new global functions which admit a port as one of its arguments, i.e. read\_answer\_XXX() is now read\_answer(serialXXX)
		\item The VelocityZ() function has been removed from source code as the Z motor velocity is controlled by the feedback loop and the motor initialization routine.
		\item StepZ members in the GUI have also been deprecated for the same reasons.
	\end{itemize}
	\item The \textit{XRayTable} external program has been implemented, the respective button in the main window should launch a pop-up with a table of the X-ray energies.
\end{enumerate}

\section{XRFMultidet\_v10.1}
\textit{Changelog with respect to version 10.0}

\begin{enumerate}
	\item Source code changes to ADCXRF.c in the Digitizer\_USB project
	\begin{itemize}
		\item A random number generator function \textit{ranqd1()} has been implemented in the ADCXRF\_USB data acquisition protocol --- now the multidetector software randomly allocates events from larger non-unitary bins into smaller unitary bins, making independent histogram addition possible.
		\item A seed function for \textit{ranqd1()} has been implemented, it calls for the system time to generate a seed after every 50E3 randomly generated numbers.
	\end{itemize}
	\item Various minor syntax bugs inherited from version 10.0 have been addressed.
\end{enumerate}


\section{XRFMultidet\_v10.2}
\textit{Changelog with respect to version 10.1}

\begin{enumerate}
	\item In project \textit{Spectrum.pro},
	\begin{itemize}
		\item Modified the value of \textit{ArraySize} variable to 16385, to avoid undefined behavior.
		\item Added a shared memory array declaration for *(shared\_memory+30000+i) dedicated to the multidetector sum.
	\end{itemize}
	\item In \textit{mainwindow\_mouse.cpp} sourec file,
	\begin{itemize}
		\item Added a conditional statement in the pixel selection switch to pass the spectrum to the appropriate shared memory array, depending on the acquisition detector (or sum) chosen in the acquisition menu. E.g., if detector A is chosen, the values found from the rectangle selection in the main window map are passed to the shared memory array for detector A.
		\item The previous allows for visualization of summed spectra obtained directly from pixel selection in the main window map.
		\item Added the \textit{ranqd1()} protocol in ADCXRF.c to permit the addition of spectra in point acquisition mode.
	\end{itemize}
	\item In ADCXRF.c, the required architecture to automatically sum the spectra and pass them to the dedicated shared memory array has been added, i.e., now every acquisition writes the spectra sum automatically to the array in *(shared\_memory+30000+i)
\end{enumerate}

\section{XRFMultidet\_v10.3}
\textit{Changelog with respect to version 10.2}

\begin{enumerate}
	\item The protocol to save .txt files in mainwindow.cpp has been modified.
	\item In project \textit{Spectrum.pro},
	\begin{itemize}
		\item The protocol to load new .txr files for both point and scan mode acquisitions has been modified to match the format required for the multidetector sum protocol
		\item The mode of display of the XRF spectra has been modified to show a histogram as opposed to connected lines.
		\item The color of the plot display has been changed from light blue against a dark blue background, to black against white with a red marker to facilitate visualization of the data.
		\item A new \textit{QLineEdit} member has been added to the bottom dock to display the name of the currently loaded file.
		\item A refresh button has been added to the menu to avoid reloading spectra from the main window if the mode of visualization has changed.
		\item The \textit{Print} GUI button and its associated \textit{print()} function have been deprecated
		\item Global boolean variables  signals to load and set calibration parameters have been replaced with a more natural signal and slot functions approach.
		\item The following signal functions have been added as members to the \textit{Mainwindow} class: \textit{toggleLog()}, and \textit{toggleEnergy(bool active)}.
		\item Calibration parameters are now persistent, i.e. they will be automatically loaded every time the XRF spectra display window is opened.
		\item Display mode (as in Channels vs. Energy) is automatically toggled and selected if calibration parameters other than '1' for gradient and '0' for offset are found.
	\end{itemize}
	\item In \textit{SHM\_Creator.cpp} and \textit{Shm.h}, the size of the shared memory segment with \textit{shmid} identifier has been updated from a size of 50 system pages (204800 bytes) to 100 pages (409600 bytes) to avoid overlap between the spectra of the second detector and the detector sum. Overall, the total stack memory usage of this segment is under 0.5 Mbytes.
	\item The *(shared\_memory+30000+i) array is now *(shared\_memory+40000+i) to avoid overlap between spectra of different detectors.
	\item In \textit{laser.cpp} source file,
	\begin{itemize}
		\item The \textit{autofocus\_timer} is now connected directly to the \textit{readKeyence()} void function through its timeout signal, avoiding the need to call for a proxy function.
		\item The structure of \textit{readKeyence()} has changed to account for the update rate from the Keyence serial, now no average is taken and if the Keyence serial returns '0'm the function awaits for the next update befor passing onto the tracking function.
		\item A port flushing protocol has been added for the Arduino port, to guarantee up-to-date distance tracking values.
		\item The interval for the \textit{autofocus\_timer} has changed to 500 ms.
	\end{itemize}
\end{enumerate}


\section{XRFMultidet\_v10.4}
\textit{Changelog with respect to version 10.3}

In \textit{mainwindow\_loadSHM.cpp} source file, 
\begin{itemize}
	\item{The body of \textit{LoadFileWithCorrection\_SHM()} is now empty as the function cannot be called from anywhere in the source code.}
	\item{The \textit{randqd1()} random number generator function has been updated to take type double parameters for its lower and upper limits, some static\_cast<type> rules have been added to guarantee proper data type conversion, and a new protocol has been added to round up or down the double type variable returned.}
	\item{The calibration gradients and offsets are calculated at the beginning of the \textit{LoadFileWithCorrection\_SHM()} as doubles by type converting the parameters stored in shared memory.} 
	\item{Lower and upper limits passed as parameters to \textit{ranqd1()} are calculated in \textit{LoadFileWithCorrection\_SHM} from the local type long \textit{dataread} by typecasting it to a double first.}
\end{itemize}


\section{XRFMultidet\_v10.5}
\textit{Changelog with respect to version 10.4}
% int setresolution() has been deprecated
% Protocols for low and high resolutions deleted from the GUI_CREATOR.cpp source file
% The Laser button, and the Helium button have been remove together with their associated widgets, mainly the divisory lines
%  and their respective text fields.
% XRFMultidet_v10.5
% In the mainwindow_loadSHM.cpp, definitions have been added to avoid artificies in the data, in particular;
%      The ranqd1() has been modified...



\section{XRFMultidet\_v10.6}
\textit{Changelog with respect to version 10.5}

The acquisition protocol defined in \textit{ADCXRF.c} has been modified as follows,
\begin{itemize}
	\item The ranqd1() function has been modified to take two float parameters as the low and high limits for random number distribution
	\item A protocol for rounding floats has been added to the ranqd1() function to avoid data truncation
	\item Where appropriate data type conversions have been made explicit
	\item The loops that allow to call the parameters from the custom DPP-GUI have been enabled and modified as appropriate
	\item The switch statement to select the protocol to write the motor's position has been moved immediately after the check for \textit{writeposition}, avoiding delays
	\item A \textit{continue} statement has been added in the energy write loop in case the buffer is empty, avoiding non-sense data being passed to shared memory segments
	\item Several, but minor, style changes favoring the use of operators for data manipulation, rather than lengthy or explicit expressions
	\item The multidetector calibration gradient and offset are calculated first, and only once, using the correct data type conversions.
\end{itemize}

\section{XRFMultidet\_v10.7}
\textit{Changelog with respect to version 10.6}

The custom graphical user interface for configuring DPP acquisition parameters is now implemented as a separate class in the \textit{XRF\_Scanner\_Dev\_3} project. The class is accessed as a pointer, and the shared memory definitions are now up to date, effectively eliminating the need for an external program to call upon this DPP GUI.

\section{XRFMultidet\_v10.8}
\textit{Changelog with respect to version 10.7}

In the \textit{Spectrum} visualization project, the following changes have been made,
\begin{itemize}
	\item The \textit{Calibration and DAQ} dock has been moved to the right, and the widget layout rules have been updated.
	\item If the channels and corresponding energies have been specified before in the calibration dialogue, in any given session, the program stores these values in a global variable and displays them as the default value in the dialogue spinboxes the next time the calibration window is opened.
	\item The calibration gradient and offset values are automatically loaded at the start
	\item The four different functions used to pass the current values in the spinboxes of the calibration dialogue to a global variable have been replaced by a single slot function which uses the \textit{QSignalMapper} class to identify the signal sender with an ID integer and customize the function's action to this ID value.
	\item The autoscale bug ---wherein the 'Events Count' scale would fluctuate apparently stochastically when the \textit{Autoscale} checkbox was active--- has been 'fixed' by ignoring the first entry of the \textit{channels} array used in the plotting of the histogram.
\end{itemize}

\section{XRFMultidet\_v10.8}
\textit{Changelog with respect to version 10.7}

In the \textit{Spectrum} visualization project, the following changes have been made,
\begin{itemize}
	\item The \textit{Calibration and DAQ} dock has been moved to the right, and the widget layout rules have been updated.
	\item If the channels and corresponding energies have been specified before in the calibration dialogue, in any given session, the program stores these values in a global variable and displays them as the default value in the dialogue spinboxes the next time the calibration window is opened.
	\item The calibration gradient and offset values are automatically loaded at the start
	\item The four different functions used to pass the current values in the spinboxes of the calibration dialogue to a global variable have been replaced by a single slot function which uses the \textit{QSignalMapper} class to identify the signal sender with an ID integer and customize the function's action to this ID value.
	\item The autoscale bug ---wherein the 'Events Count' scale would fluctuate apparently stochastically when the \textit{Autoscale} checkbox was active--- has been 'fixed' by ignoring the first entry of the \textit{channels} array used in the plotting of the histogram.
\end{itemize}

\section{XRFMultidet\_v10.9}
\textit{Changelog with respect to version 10.8}

\begin{itemize}
	\item An array has been introduced in SHM\_CREATOR.cpp to store the shmIDs for shared memory segments
	\item A template function has been introduced to optimize the source code, now shared memory segments are attached by simply calling this function with the desired \textbf{key\_t} \textit{key} and specifying the desired data type for this SHM segment.
	\item The program exit protocol has been moved from the class destructor to a virtual slot function which responds to program closure signals; the \textit{closeeventsignal()} function only accepts the role of the closure signal once all relevant processes have been killed and the shared memory segments been both called for erasure and dettached.
	\item Other minor GUI modifications
\end{itemize}

\section{XRFMultidet\_v11.0}
\textit{Changelog with respect to version 10.9}

The protocol for random addition of spectra has been added to the function which renders the composed element (RGB) map. The modifications are analogue to those made to the function which renders a map for a single specified energy (or channel) interval, and listed in the changelog for XRFMultidet\_v10.5

\section{XRFMultidet\_v11.1}
\textit{Changelog with respect to version 11.0}

\begin{itemize}
	\item PixelCrct() function has been deprecated
	\item SelectionElement() functions have been removed and replaced by the single function writeCompMapLimits() which uses the QSignalMapper class to identify the sender of a signal, and assigns their value to the respective global variables
	\item The LoadElementsmapSum() function has been modified so as to produce a dialog with QSpinBoxes as opposed to QComboBoxes, now the limits for the composed map are inputted manually as opposed to being loaded from an outdated .txt file
	\item Other aesthetic modifications for this dialog box such as a clearer instruction message and a different layout
	\item MapCorrection() and CutPixels() functions have been deprecated
	\item Data type changed from int to double for composed map element limits
\end{itemize}


\end{document}


